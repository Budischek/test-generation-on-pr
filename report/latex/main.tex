\documentclass[12pt, a4paper]{article}

% ********************************************************
%				        PREAMBLE
% ********************************************************

\newcommand{\dd}[1]{\mathrm{d}#1}             % For integral design
\newcommand{\HRule}{\rule{\linewidth}{0.5mm}} % Defines a new command for the horizontal lines, change thickness here

% --------------------------------------------------------
%	             Set page Size and Margins
% --------------------------------------------------------

\usepackage[a4paper,top=3cm,bottom=3cm,left=3cm,right=3cm,marginparwidth=1.75cm]{geometry}

% --------------------------------------------------------
%	           Language and Font Encodings
% --------------------------------------------------------

\usepackage[T1]{fontenc}     % Includes more chars
\usepackage[utf8]{inputenc}  % Includes more chars
\usepackage{kotex}           % For Korean/Hangul
%\usepackage[swedish]{babel} % For åäö

%--------------------------------------------------------
%	                  Bibliography
%--------------------------------------------------------

\usepackage[style=ieee]{biblatex} % Change style of Bibliography, currently IEEE
\addbibresource{references.bib}   % Name of .bib file

% --------------------------------------------------------
%		  			 Useful Packages
% --------------------------------------------------------

\usepackage{graphicx}	   % To import images
\usepackage{float}         % Stop figures from floating
\usepackage[]{algorithm2e} % To include algorithms
\usepackage{mathtools}	   % To use mathematical expressions, includes {amsmath}
\usepackage{amsmath}
\usepackage{array,tabularx}% To have arrays. Used for equation variable def
\usepackage{parskip}       % Return in code, render on preview too
\usepackage{verbatim}      % To do \beginEnd{comment}
\usepackage{xcolor}        % To include xcolor
%\usepackage{pdfpages} 	   % usage: \includepdf{file.pdf}

% --------------------------------------------------------
%	           	        Hyper Links
% --------------------------------------------------------

\usepackage{url}		          % For links
\usepackage[hidelinks]{hyperref}  % Clickable links
\hypersetup{ % ^[hidelinks] to remove ugly boxes
	colorlinks = true,            % Coloured links instead of ugly boxes
  	urlcolor   = {blue!50!black}, % Colour for external hyperlinks
  	linkcolor  = {blue!50!black}, % Colour of internal links
  	citecolor  = {red!50!black}   % Colour of citations
}

% --------------------------------------------------------
%	               Custom Environments
% --------------------------------------------------------

\newenvironment{conditions} % for equation variable definition
  {\par\vspace{0pt}\noindent
   \tabularx{\columnwidth}{>{$}l<{$} @{}>{${}}c<{{}$}@{} >{\raggedright\arraybackslash}X}}
  {\endtabularx\par\vspace{\belowdisplayskip}}

% to have subsections given in letters: 1.a, 1.b etc..
% \renewcommand{\thesubsection}{\thesection.\alph{subsection}}

% --------------------------------------------------------
%	               Setup Code Environment
% --------------------------------------------------------

\usepackage{listings}
\usepackage{color}
\definecolor{mygreen}{rgb}{0,0.6,0}
\definecolor{mygray}{rgb}{0.5,0.5,0.5}
\definecolor{mymauve}{rgb}{0.58,0,0.82}

\lstset{
  backgroundcolor=\color{white},   % choose the background color; you must add \usepackage{color} or \usepackage{xcolor}; should come as last argument
  basicstyle=\tiny,        % the size of the fonts that are used for the code
  breakatwhitespace=false,         % sets if automatic breaks should only happen at whitespace
  breaklines=true,                 % sets automatic line breaking
  captionpos=b,                    % sets the caption-position to bottom
  commentstyle=\color{mygreen},    % comment style
  deletekeywords={...},            % if you want to delete keywords from the given language
  escapeinside={\%*}{*)},          % if you want to add LaTeX within your code
  extendedchars=true,              % lets you use non-ASCII characters; for 8-bits encodings only, does not work with UTF-8
  frame=single,	                   % adds a frame around the code
  keepspaces=true,                 % keeps spaces in text, useful for keeping indentation of code (possibly needs columns=flexible)
  keywordstyle=\color{blue},       % keyword style
  language=Octave,                 % the language of the code
  morekeywords={*,...},            % if you want to add more keywords to the set
  numbers=left,                    % where to put the line-numbers; possible values are (none, left, right)
  numbersep=5pt,                   % how far the line-numbers are from the code
  numberstyle=\tiny\color{mygray}, % the style that is used for the line-numbers
  rulecolor=\color{black},         % if not set, the frame-color may be changed on line-breaks within not-black text (e.g. comments (green here))
  showspaces=false,                % show spaces everywhere adding particular underscores; it overrides 'showstringspaces'
  showstringspaces=false,          % underline spaces within strings only
  showtabs=false,                  % show tabs within strings adding particular underscores
  stepnumber=2,                    % the step between two line-numbers. If it's 1, each line will be numbered
  stringstyle=\color{mymauve},     % string literal style
  tabsize=2,	                   % sets default tabsize to 2 spaces
  title=\lstname                   % show the filename of files included with \lstinputlisting; also try caption instead of title
}

% --------------------------------------------------------
%	                 Title and Author
% --------------------------------------------------------

\title{Research Paper Template} % Title of the document
\author{Gustav Janér}  % Author. Do not change
% \maketitle % Prints title and author

% ********************************************************
%				      DOCUMENT BEGIN
% ********************************************************

\begin{document}
\begin{titlepage} 
\center 	     % Center everything on the page

% --------------------------------------------------------
%	                 Heading Sections
% --------------------------------------------------------

\textsc{\LARGE Artificial Intelligence Based Software Engineering} \\[1cm]
% Major heading such as course name

\textsc{\Large Project Report}\\[0.5cm]
% Minor heading such as course title

% --------------------------------------------------------
%	                   Title Section
% --------------------------------------------------------

\HRule \\[0.4cm]
{ \huge \bfseries Continuous Test Generation on Pull Request}\\[0.03cm] % Title of the document
\HRule \\[1.5cm]
 
% --------------------------------------------------------
%	               Date and Names Section
% --------------------------------------------------------

{\Large \underline{Group 5} \\
Mathias Sixten \\
Yixuan  \\
David Budischek, 201863  \\
Gaspard \\
Gustav Janér, 20186416 \\
} % Authors of the document

\vspace{0.3cm}
\today
\vspace{4cm}

% --------------------------------------------------------
%	                   Logo Section
% --------------------------------------------------------

\includegraphics[scale=0.15]{KAIST.png}\\[1cm]

% --------------------------------------------------------

\vfill % Fill the rest of the page with whitespace
\end{titlepage}

% ********************************************************
%		  		          MAIN
% ********************************************************

\section*{Abstract}
-

\section{Introduction}
Over the past decades software release cycles have shortened repeatedly. Two decades ago it was common place to release software and then maybe follow up with a patch (in the form of a physical hard disk) a few years down the line. From that we went to pushing out monthly, or even weekly, security updates. In recent years though, it became common place to not bundle releases anymore. Instead every single feature/bug/typo fix is published instantaneously and pushed to every user. With Continuous Delivery come a plethora of challenges. One of which is testing. In a traditional release cycle there is enough time to hand over the release candidate to QA and give them time to verify functionality. But this won't do nowadays. Instead we rely on automated tests to quickly verify functionality. In practice, tests can never guarantee a bugfree product, instead they can only try and cover as much of the code as possible. 

In addition to manually writing tests there are advanced unit test generation tools available (such as EvoSuite or Pex). While these tools require no manual input, they do require a lot of computation time. In their 2014 paper \cite{campos_continuous_2014} the authors of EvoSuite propose a solution to this issue, Continuous Test Generation. Instead of a greenfield test generation for each new iteration, they leverage previous results to speed up the test generation process. In their research this is done by smartly allocating resources to the classes that changed or are not yet fully covered.

When it comes to integrating CTG in the CI/CD workflow commonly used in modern Software projects there is little literature available. In a case study done as part of the SSBSE'15 challenge \cite{barros_continuous_2015} we can clearly see that CTG performs well enough to warrant implementation. Due to this we implemented an out of the box, platform agnostic tool to easily add CTG to an existing project without interfering with the infrastructure that has previously been set up.

\section{Body}


\section{Results}


\section{Conclusion}


% --------------------------------------------------------
%	                   Bibliography
% --------------------------------------------------------

\newpage
\addcontentsline{toc}{section}{References} % Renders References appear in the Table of Content
\printbibliography % Prints the actual Bibliography

% --------------------------------------------------------
%	                     Appendix
% --------------------------------------------------------

% \newpage
% \appendix

%\section{Appendix} \label{sec:appx}

\end{document}
